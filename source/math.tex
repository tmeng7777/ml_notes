%%%%%%%%%%%%%%%%%%%%%%%%%%%%%%%%%%%%%%%%%%%%%%%%%%%%%%%%%%%%%%%%
%
%  Template for creating scribe notes for cos511.
%
%  Fill in your name, lecture number, lecture date and body
%  of scribe notes as indicated below.
%
%%%%%%%%%%%%%%%%%%%%%%%%%%%%%%%%%%%%%%%%%%%%%%%%%%%%%%%%%%%%%%%%


\documentclass[14pt]{article}

\setlength{\topmargin}{0pt}
\setlength{\textheight}{12in}
\setlength{\headheight}{0pt}
\setlength{\headsep}{0pt}
\setlength{\oddsidemargin}{0.25in}
\setlength{\textwidth}{6in}
\pagestyle{plain}

\begin{document}

\thispagestyle{empty}


\begin{center}
\bf\large Machine Learning Notes
\end{center}

\noindent
Author: Tao Meng  %%% FILL IN LECTURER (if not RS)
\hfill
Series\ One              %%% FILL IN LECTURE NUMBER HERE
\\
           %%% FILL IN YOUR NAME HERE
\hfill
11/13/2018                         %%% FILL IN LECTURE DATE HERE

\noindent
\rule{\textwidth}{1pt}

\medskip

%%%%%%%%%%%%%%%%%%%%%%%%%%%%%%%%%%%%%%%%%%%%%%%%%%%%%%%%%%%%%%%%
%% BODY OF SCRIBE NOTES GOES HERE
%%%%%%%%%%%%%%%%%%%%%%%%%%%%%%%%%%%%%%%%%%%%%%%%%%%%%%%%%%%%%%%%

\section*{Statistics}
\begin{flushleft}
Maximum likelihood estimation is an important statistical approach to estimate parameters. Here is one example of deriving the
mean and variance of the data forming Gaussian Distribution. This problem was tested: \par
\ \par
The probability density function (pdf) for a 1-d Gaussian Distribution is as follows:\par\par
\begin{equation}
f(x | \mu,\sigma) = \frac{1}{{\sigma \sqrt{2\pi } }} e^{ {- \left( x - \mu \right)^2 } / { 2 \sigma^2} } 
\end{equation}









%%%%%%%%%%%%%%%%%%%%%%%%%%%%%%%%%%%%%%%%%%%%%%%%%%%%%%%%%%%%%%%%
\end{flushleft}
\end{document}
